\chapter{Ambiente de desenvolvimento}
 
\section{Interpretador perl} 
Usu\'arios de sistemas derivativos Unix, como Linux e BSD, possuem uma grande probablidade de que a sua distribui\c{c}\~ao/SO j\'a possua um interpretador 
Perl pr\'e-instalado. Aos usu\'arios Windows \'e necess\'ario instalar o ''\textit{Active Perl}'', que pode ser facilmente obtido em www.activestate.com. 
 
Caso sua distribui\c{c}\~ao por algum motivo n\~ao tiver o interpretador Perl pr\'e-instalado, \'e poss\'ivel instal\'a-lo usando algum dos comandos abaixo. 

\begin{itemize}
    \item{Ubuntu ou Debian: sudo apt-get install perl}
    \item{Arch Linux: Pacman S perl}
    \item{OpenSUSE: zypper install perl}
    \item{Fedora/Red Hat Enterprise: yum install perl}
\end{itemize}

\section{Editor de texto}  
\'E necess\'ario um editor de texto qualquer para escrever nossos c\'odigos, nessa apostila ser\'a usado o Sublime Text 3 
(dispon\'ivel via www.sublimetext.com/3) e o vim. 